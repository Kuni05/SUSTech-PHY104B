\documentclass[12pt,a4paper]{article}

% 如果需要中文支持,推荐使用xeCJK + 字体设置
\usepackage{xeCJK}
\setCJKmainfont[BoldFont=SimHei]{SimSun}  % 示例:宋体,可根据系统字体情况更换
\usepackage{amsmath}      % 数学公式(如有需要)
\usepackage{graphicx}     % 插图
\usepackage{geometry}     % 调整页边距
\usepackage{fancyhdr}     % 自定义页眉页脚
\usepackage{indentfirst}  % 中文首行缩进
\usepackage{calc}         % 允许做长度运算(测量文字宽度等)
\usepackage{titlesec}
\usepackage{setspace}

% 设置 \section 级标题为:加粗、大字号(如 \Large)
\titleformat{\section}
	{\bfseries\large}    % 标题自身的格式
	{\thesection}        % 标题编号的显示方式
	{1em}                % 编号与标题文字之间的间距
	{}                   % 在标题文字前后可插入额外代码,此处为空
	
% 设置 \subsection 级标题为:加粗、中等字号(如 \normalsize)
\titleformat{\subsection}
	{\bfseries\normalsize}
	{\thesubsection}
	{1em}
{}

% 页面设置(可根据需要微调)
\geometry{
	left=2cm,
	right=2cm,
	top=1cm,
	bottom=2cm
}

% 不需要过大的行距,使用较接近单倍行距的设置
\renewcommand{\baselinestretch}{1.1}

% 仅在页脚居中显示页码,页眉保持为空
\pagestyle{fancy}
\fancyhf{}  % 清空默认的页眉页脚
\fancyfoot[C]{\thepage}
\renewcommand{\headrulewidth}{0pt}
\renewcommand{\footrulewidth}{0pt}

% 首行缩进2字符(中文习惯)
\setlength{\parindent}{0pt}
\setlength{\leftskip}{2em}

\begin{document}
	\doublespacing
	
	
	%-------------------------------------------------------
	% 1) 并排两个minipage:左标题、右校徽
	%   - 0.65\textwidth + 0.35\textwidth = \textwidth
	%   - 如果校徽过大或过小,可改宽度,如 0.25\textwidth、0.3\textwidth 等
	%   - 如果想让标题更大,可将 \Huge 改成 \huge 或 \LARGE
	%-------------------------------------------------------
	\noindent
	\hspace{-2em}
	\begin{minipage}[c]{0.65\textwidth}
		\raggedright
		{\fontsize{40pt}{60pt}\selectfont 物理实验报告}
	\end{minipage}
	\begin{minipage}[c]{0.35\textwidth}
		\raggedleft
		% 强制把校徽拉大到 0.35\textwidth 宽度 (高度自动匹配)
		% 若想指定高度,可用 "height=3cm" 等. 二选一即可.
		\includegraphics[width=\linewidth, trim={20cm 20cm 20cm 20cm}, clip]{university_logo.png}
	\end{minipage}

	\vspace{-1em}
	

	%下方画两条分割线,并在两线之间写学号、姓名、日期、时间
	
	\hrule
	\vspace{0.4em}
	\hspace{-2.2em}
	\noindent
	学号:\underline{\makebox[2.7cm][c]{114514}} \quad
	姓名:\underline{\makebox[2.7cm][c]{SUSTech}} \quad
	日期:\underline{\makebox[2.7cm][c]{年/月/日}} \quad
	时间:\underline{\makebox[2.7cm][c]{周二下午班}}
	\vspace{-1em}
	\par
	\hrule
	

	%正文示例

	
	\section{实验名称:单摆测量重力加速度}
	
	\section{实验目的}
	\par
	1) 利用经典的单摆公式,依据器材和对重力加速度的测量精度要求,进行设计性实验基本方法的演练。 
	\\
	2) 学习应用误差均分原则,选用适当的仪器和测量方法,完成设计性实验内容。
	
	\section{实验仪器}
	游标卡尺,钢卷尺,电子秒表,单摆实验仪
	
	\section{实验原理}
	
	\textbf{1)}已知单摆周期的理论公式为:\[ T = 2\pi \sqrt{\frac{l}{g}} \]
	
	因此我们能推导出重力加速度的计算公式:\[ g = \frac{4\pi^2 l}{T^2} \]
	其中\( l \)为摆长,\( T \)为单摆周期,\( t \)为测量时间,\( N \)为周期数。	
	
	由于摆长\( l = \frac{D}{2} + l_{\text{绳长}} \),
	
	因此待测物理量有:总时间\( t \),小球直径\( D \),绳长\( l_{\text{绳长}} \)。
	\\
	\\
	\textbf{2)}根据相对误差要求\( \frac{\Delta g}{g} < 1\% \),以及误差均分原理\( \frac{\Delta g}{g} = \frac{\Delta l}{l} + \frac{2\Delta t}{t} \),则有:\[ \frac{\Delta l}{l} < 0.5\% , \frac{2\Delta t}{t} < 0.5\%  \]
	
	\textbf{对\( \frac{\Delta l}{l} \)进行估算},假设摆长\( \approx \)70.00cm,为了达到\(\frac{\Delta l}{l} < 0.5\%  \),则\( \Delta l < 0.35cm \)
	
	根据\( \Delta l = \frac{\Delta D}{2} + \Delta l_{\text{绳长}} \),利用仪器的最大允差进行估算,可得:
	\[ \Delta l_{\text{绳长}} = \Delta \text{米} \approx 0.08cm \]
	\[ \Delta D = \Delta \text{卡尺} \approx 0.002cm \]
	\[ \Delta l = \frac{\Delta D}{2} + \Delta l_{\text{绳长}} \approx 0.081cm << 0.35cm \]
	因此,实验使用钢卷尺测量线长,游标卡尺测量小球直径,可以满足\(\frac{\Delta l}{l} < 0.5\%  \)。
	
	\textbf{对\( \frac{2\Delta t}{t} \)进行估算},秒表精度$\Delta \text{秒} \approx 0.01s$,开停秒表的总反应时间$\Delta \text{人} \approx 0.2s$,则:\[ \Delta t = \Delta \text{秒} + \Delta \text{人} \approx 0.2s \]
	假设单摆周期$T = 1.7s$,为保证\( \frac{2\Delta t}{t} < 0.5 \% \),利用$t = NT$得$N>47$,因此可以通过测量多个单摆周期来减小时间测量的误差,提高周期测量的精确度。
	\\
	\\
	此处我们得出,至少要测量48个单摆周期才能达到测量精度的要求。
	
	
	\section{实验内容}
	根据要求设计实验,步骤如下:
	
	(1)使用游标卡尺测量小球直径$D$。
	
	(2)将摆线挂上单摆实验仪,使用钢卷尺测量摆线长度$l_{\text{绳长}}$。调整夹具,控制绳长在70.00cm附近。
		
	(3)将小球拉开一个小于5度的摆角并释放,小球相邻两次向右经过摆线和中线重合点(及最低点)之间的时间间隔记为一个周期。
	
	(4)使用秒表测量$N$次周期的时长$t$,$N$取50(依据见实验原理部分)。小球第1次经过最低点时启动秒表,第51次经过最低点时按停秒表。
	
	(5)每个物理量重复测量五次并取平均值以减小随机误差。
	
	(6)根据公式处理数据,得出结论。
	
	\section{数据记录}
	\begin{center}
		\begin{tabular}{|c|c|c|c|c|c|c|}
		\hline
		 & 1 & 2 & 3 & 4 & 5 & 平均值 \\
		\hline
		摆线长度$l_{\text{绳长}}$(cm) & 70.52 & 70.75 & 71.11 & 70.95 & 70.90 & 70.846 \\
		\hline
		小球直径$D$(mm) & 20.00 & 20.04 & 20.02 & 20.02 & 20.00 & 20.016 \\
		\hline
		\end{tabular}
	\end{center}
	
	\begin{center}
		\begin{tabular}{|c|c|c|c|c|c|c|}
			\hline
			& 1 & 2 & 3 & 4 & 5 & 平均值 \\
			\hline
			50个单摆周期$t$(s) & 85.00 & 85.09 & 85.06 & 85.06 & 85.03 & 85.048 \\
			\hline
		\end{tabular}
	\end{center}
	后附原始数据。
	
	\section{数据处理}
	根据公式:\[ g = \frac{4\pi^2 l}{T^2} \]
	代入数据计算:\[ g = \frac{4\pi^2 \times 50^2 \times (0.70846 + \frac{0.02016}{2}) } {85.048^2} \]
	得:\[ g \approx 9.80443 m/s^2  \]
	\textbf{计算A类不确定度:}
	\[ u_a(l_{\text{绳长}}) = \sqrt{\frac{\sum(l_{\text{绳长}i} - \bar{l}_{\text{绳长}})^2}{n(n-1)}} = \sqrt{\frac{(70.52 - 70.846)^2 + \cdots + (70.90 - 70.846)^2}{5 \times 4}} \approx 0.09973cm \] 
	
	\[ u_a(D) = \sqrt{\frac{\sum(D_i - \bar{D})^2}{n(n-1)}} = \sqrt{\frac{(20.00 - 20.016)^2 + \cdots + (20.00 - 20.016)^2}{5 \times 4}} \approx 0.007483mm \]
	
	\[ u_a(t) = \sqrt{\frac{\sum(t_i - \bar{t})^2}{n(n-1)}} = \sqrt{\frac{(85.00 - 85.048)^2 + \cdots + (85.03 - 85.048)^2}{5 \times 4}} \approx 0.01530s \]
	
	\textbf{计算B类不确定度:}
	\[ u_b(l_{\text{绳长}}) = \frac{\sqrt{\Delta^{2}_{\text{估}}(l_{\text{绳长}}) + \Delta^{2}_{\text{仪}}(l_{\text{绳长}})}}{C_\text{米}}  = \frac{\sqrt{0.05^2 + 0.08^2}}{3} \approx 0.03144cm \]
	
	\[ u_b(D) = \frac{\sqrt{\Delta^{2}_{\text{估}}(D) + \Delta^{2}_{\text{仪}}(D)}}{C_\text{卡尺}}  = \frac{\sqrt{0.02^2 + 0.02^2}}{\sqrt{3}} \approx 0.01633mm \]
	
	\[ u_b(t) = \frac{\sqrt{\Delta^{2}_{\text{估}}(t) + \Delta^{2}_{\text{仪}}(t)}}{C_\text{秒}}  = \frac{\sqrt{0.2^2 + 0.01^2}}{3} \approx 0.06675s \]
	
	\textbf{计算各物理量取置信区间为0.95的展伸不确定度:}
	
	$P=0.95$,公式为:
	\[ U_{0.95} = \sqrt{(t_{0.95}u_a)^2 + (k_{0.95}u_b)^2} \]
	对于$n=5,P=0.95$,查表得$t_{0.95}=2.78$。
	
	钢卷尺,电子秒表遵循正态分布,$k_{0.95} = 1.96$。
	
	游标卡尺遵循均匀分布,$k_{0.95} = 1.65$。
	\[ U_{0.95}(l_{\text{绳长}}) = \sqrt{(2.78\times 0.0997296)^2 + (1.96\times 0.03144)^2} \approx 0.1720cm  \]
	
	\[ U_{0.95}(D) = \sqrt{(2.78\times 0.007483)^2 + (1.64\times 0.01633)^2} \approx 0.02435mm  \]
	
	\[ U_{0.95}(t) = \sqrt{(2.78\times 0.01530)^2 + (1.96\times 0.06675)^2} \approx 0.09687s  \]
	
	\textbf{使用不确定度传递公式,合成间接测得的重力加速度的不确定度:}
	
	由于\( l = \frac{D}{2} + l_{\text{绳长}} \),有
	
	\[ U_{0.95}(l) =  \sqrt{U_{0.95}^2(l_{\text{绳长}}) + (\frac{1}{2}U_{0.95}(D))^2} = \sqrt{1.720^2 + (\frac{1}{2}\times 0.02435)^2} \approx 1.720mm \]
	
	又有:
	
	\[ \frac{U_{0.95}(g)}{g} = 	\sqrt{(\frac{U_{0.95}(l)}{l})^2 + (\frac{2U_{0.95}(t)}{t})^2} = \sqrt{(\frac{1.720}{718.47})^2 + (\frac{2\times 0.09687}{85.048})^2} \approx 0.003305 \]
	
	\[ U_{0.95}(g) = 0.003305 \times 9.80443 \approx 0.03240m/s^2 \]
	
	\textbf{综上,实验测得的重力加速度为$(9.80443 \pm 0.03240)m/s^2(P=0.95)$}.
	
	使用深圳本地重力加速度参考值$g = 9.7887m/s^2$,可计算测量值的相对误差:
	\[ \frac{\Delta g}{g} = \frac{9.80443-9.7887}{9.7887} \approx 0.0016 = 0.16\% < 1\% \]
	测量结果符合设计要求。
	
	\section{误差分析}
	(1)将摆角小于5度的单摆运动近似为简谐运动,近似过程存在误差。
	\\
	(2)操作时很难让单摆做纯粹的平面内摆动,几乎必然演变成圆锥摆。
	\\
	(3)摆线和单摆实验仪连接处存在摩擦力,小球摆动受空气阻力。
	\\
	(4)摆线有一定弹性,在小球摆动时受向心力会略微变长。
	
	\section{实验结论}
	本实验利用单摆实验仪以及一系列测量工具,通过计算间接测得本地重力加速度。
	\\
	测量值$g=(9.80443 \pm 0.03240)m/s^2(P=0.95)$。
	\\
	相对误差$\frac{\Delta g}{g} = 0.16\%$,符合设计要求。

	
	
\end{document}
